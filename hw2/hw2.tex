\documentclass{article}
\linespread{1.3}
\usepackage[margin=50pt]{geometry}
\usepackage{amsmath, amsthm, amssymb, amsthm, tikz, fancyhdr, graphicx}
\pagestyle{fancy}
\renewcommand{\headrulewidth}{0pt}
\newcommand{\changefont}{\fontsize{15}{15}\selectfont}

\fancypagestyle{firstpageheader}
{
  \fancyhead[R]{\changefont Michael Huang \\ CFRM 420 \\ Homework 2}
}

\begin{document}

\thispagestyle{firstpageheader}

\section*{1.}
{\Large 

\subsection*{(a)}

We aim to find $\mathbb{P}(X \in \{3, 4, 5\})$ with $X \sim \text{Poisson}(2)$, or the probability that $X$ is 3, 4, or 5 within the Poisson distribution with $\lambda = 2$. To do this, we can find the respective pdfs and sum them up: \\
$\mathbb{P}(X = 3) = 0.18044704$ \\
$\mathbb{P}(X = 4) = 0.09022352$ \\
$\mathbb{P}(X = 5) = 0.03608941$ \\ 
$\mathbb{P}(X \in \{3, 4, 5\}) = 0.18044704 + 0.09022352 + 0.03608941 = $ \framebox[1.1\width]{\textbf{0.30675997}}

\subsection*{(b)}

We aim to find $\mathbb{P}(Z < -4)$ with $Z \sim \text{N}(0, 1)$. This is the default normal, and we essentially just need to take the cdf of -4, which we can look up / calculate in R to be \framebox[1.1\width]{\textbf{3.167124e-05}}

\subsection*{(c)}

$\mathbb{P}(T / a < -4)$ with $T \sim \text{t}(10)$ and $a = 1.25$; that is, $T$ has a Student's $t$ distribution with 10 degrees of freedom. Again, we take the cdf of -4 but multiply by the $a$ term of 1.25, which we can look up / calculate in R to be \framebox[1.1\width]{\textbf{0.0002686668}}.

\subsection*{(d)}

The likelihood of a 4-sigma event is much higher for the heavy-tailed distribution than the normal distribution ( 0.0002686668 / 3.167124e-05 = 8.48299 times as likely); we know that the heavy-tailed t-distribution has higher kurtosis and heavier distribution towards the tails (as is the name) than the normal distribution. We can therefore say that the t-distribution will do a better job modeling more extreme events, like the 4-sigma event, where the returns are 4 standard deviations below the current value, when it comes to modeling extreme events for asset returns. 

}

\section*{2.}
{\Large

We model the one-day log return for a stock is $t_{t+1} \sim N(0.0006, 0.02^2)$, and the current investment $P_t = \$ 10$ million.

\subsection*{(a)}

We aim to find the value $x$ such that the probability that the investor suffers a one-day log return less than $x$ is $1\%$. Essentially, we want to find $\mathbb{P}(X < x) = 0.01$. We can do this by using the quantile function using the normal as described; calculating this using R we can find this log return value to be \framebox[1.1\width]{\textbf{-0.04592696}}


\subsection*{(b)}

We aim to find the 99\% value-at-risk, which we can do by calculating $-(P_{t+1} - P_t)$. We first calculate $P_{t+1}$. We know log returns is calculated such that \\
$-0.04592696 = \text{ln}(\frac{P_{t+1}}{P_t})$ \\
$e^{-0.04592696} = \frac{P_{t+1}}{P_t}$ \\ 
$e^{-0.04592696} = \frac{P_{t+1}}{10 \text{ million}}$ \\ 
$P_{t+1} = 9.551117 \text{ million}$ \\ \\
$99\% \text{ VAR} = -(P_{t+1} - P_t) = -(9.551117 - 10) \text{ million} = 0.448883 \text{ million} = $ \framebox[1.1\width]{\textbf{\$448,883}}

}

\section*{3.}
{\Large 

\subsection*{(a)}

We can show that $\Sigma$ is a valid covariance matrix if it is positive semidefinite. To do this, we can equivalently show that the eigenvalues are greater than or equal to zero. Doing this in R, we see that the eigenvalues for $\Sigma$ are 2.8 and 0.2, both of which are indeed greater than or equal to zero.

\subsection*{(b)}

We know that $X_1 + X_2$ would represent the total log return across the two consecutive periods, or $T_0$ to $T_2$, due to the logarithmic nature of the returns. Since $X_1$ and $X_2$ have a bivariate normal distribution, $X_1 + X_2$ would also be a normal distribution, since any linear combination of $X_1$ and $X_2$ would result in a normal distribution. Its distribution would be as follows: \\
$\mu_{X_1 + X_2} = E(X_1 + X_2) = E(X_1) + E(X_2) = 1 + 2 = 3$ \\
$\text{Var}(X_1 + X_2) = \text{Var}(X_1) + \text{Var}(X_2) + 2\text{Cov}(X_1, X_2) = 1 + 2 + 2 \cdot -1.2 = 0.6$ \\ \\ 
This is a normal distribution with mean 3 and variance 0.6, or \framebox[1.1\width]{\textbf{N(3, 0.6)}}

}

\end{document}
