\documentclass{article}
\linespread{1.3}
\usepackage[margin=50pt]{geometry}
\usepackage{amsmath, amsthm, amssymb, amsthm, tikz, fancyhdr, graphicx}
\pagestyle{fancy}
\renewcommand{\headrulewidth}{0pt}
\newcommand{\changefont}{\fontsize{15}{15}\selectfont}

\fancypagestyle{firstpageheader}
{
  \fancyhead[R]{\changefont Michael Huang \\ CFRM 420 \\ Homework 1}
}

\begin{document}

\thispagestyle{firstpageheader}

\section*{1.}
{\Large 

%\begin{verbatim}
%  Text enclosed inside \texttt{verbatim}
%  environment 
%  is printed directly 
%  and all \LaTeX{} commands are ignored.
%\end{verbatim}
%
%\framebox[1.1\width]{\textbf{answer}}

\subsection*{(a)}

We aim to find $\mathbb{P}(X \in \{3, 4, 5\})$ with $X \sim \text{Poisson}(2)$, or the probability that $X$ is 3, 4, or 5 within the Poisson distribution with $\lambda = 2$. To do this, we can find the respective pdfs and sum them up: \\
$\mathbb{P}(X = 3) = 0.18044704$ \\
$\mathbb{P}(X = 4) = 0.09022352$ \\
$\mathbb{P}(X = 5) = 0.03608941$ \\ 
$\mathbb{P}(X \in \{3, 4, 5\}) = 0.18044704 + 0.09022352 + 0.03608941 = $ \framebox[1.1\width]{\textbf{0.30675997}}

\subsection*{(b)}

We aim to find $\mathbb{P}(Z < -4)$ with $Z \sim \text{N}(0, 1)$. This is the default normal, and we essentially just need to take the cdf of -4, which we can look up / calculate in R to be \framebox[1.1\width]{\textbf{3.167124e-05}}

\subsection*{(c)}

% check and see if this is right for sd_nu adjustment or not
$\mathbb{P}(T / a < -4)$ with $T \sim \text{t}(10)$ and $a = 1.25$; that is, $T$ has a Student's $t$ distribution with 10 degrees of freedom. Again, we take the cdf of -4, which we can look up / calculate in R to be \framebox[1.1\width]{\textbf{0.001259166}}.
% or 0.0005967334

\subsection*{(d)}

The quantities calculated in (b) and (c) are known as the probabilities of a 4-sigma event. Suppose that $Z$ and $T/a$ are the log returns of two different assets. Briefly comment on how much more likely 4-sigma events are for $T/a$ compared to $Z$ and what this reveals about these asset returns. \\ \\
The likelihood of a 4-sigma event, where the log returns are 4 standard deviations below the current value, is much more likely for the heavy-tailed distribution based asset than the normal distribution based asset.

}

\section*{2.}
{\Large

Suppose the one-day log return for a stock is $t_{t+1} \sim N(0.0006, 0.02^2)$. At the current time, an investor has $P_t = \$ 10$ million invested in the stock.  

\subsection*{(a)}

Find the value $x$ such that the probability that the investor suffers a one-day log return less than $x$ is $1\%$.

\subsection*{(b)}

The $99\%$ value-at-risk is the value such that the probability the investor incurs a greater one-day loss is $1\%$, that is, $-(P_{t+1} - P_t)$, where $P_{t+1}$ is the value of the investment if the one-day log return $x$ were to happen. Find the $99\%$ value-at-risk. 

}

\section*{3.}
{\Large 

Let \textbf{X} = $(X_1, X_2 \sim \text{N}(\boldsymbol{\mu}, \Sigma))$ be a bivariate normal distribution, where \\

$\boldsymbol{\mu} = $
\begin{pmatrix}
1 \\
2	
\end{pmatrix}, 
$\Sigma=$
\begin{pmatrix}
1 & -1.2 \\
-1.2 & 2
\end{pmatrix}.

\subsection*{(a)}

Explain why $\Sigma$ is a valid covariance matrix.

\subsection*{(b)}

If $X_1$ and $X_2$ are the one-period log returns of two consecutive periods, explain what $X_1 + X_2$ represents, and find the distribution of $X_1 + X_2$.

}

\end{document}