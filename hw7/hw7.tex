\documentclass{article}
\linespread{1.3}
\usepackage[margin=50pt]{geometry}
\usepackage{amsmath, amsthm, amssymb, amsthm, tikz, fancyhdr, graphicx}
\pagestyle{fancy}
\renewcommand{\headrulewidth}{0pt}
\newcommand{\changefont}{\fontsize{15}{15}\selectfont}

\fancypagestyle{firstpageheader}
{
  \fancyhead[R]{\changefont Michael Huang \\ CFRM 420 \\ Homework 7}
}

\begin{document}

\thispagestyle{firstpageheader}

\section*{1.}
{\Large 

% \begin{verbatim}
%   Text enclosed inside \texttt{verbatim}
%   environment 
%   is printed directly 
%   and all \LaTeX{} commands are ignored.
% \end{verbatim}

% \framebox[1.1\width]{\textbf{answer}}

\subsection*{(a)}

We implement this in R. The t-test assumes that the sample comes from a normal distribution.
% but that the monthly log returns are not normal. 
% what does this mean?
\\
We get the following results: \\ 
t-statistic: 2.9423 \\
p-value: 0.006696 \\ \\
Since our p-value of $0.006696 < 0.05$, we end up rejecting the null hypothesis that $\mu_1 = \mu_2$ at a 5\% significance level; i.e. the difference in the means of the two series is statistically significant, in fact, $\mu_1$ is statistically significantly greater than $\mu_2$.

\subsection*{(b)}

We implement this in R, and get the following results: \\ 
t-statistic = -2.1296 \\
p-value = 0.05662 \\ \\ 
Since our p-value of $0.05662 > 0.05$, we end up failing to reject the null hypothesis that $\mu_2 = 0.16$ at a 5\% significance level; i.e. the difference between $\mu_2$ and $0.16$ is not statistically significant.


}

\section*{2.}
{\Large 
\subsection*{(a)}



\subsection*{(b)}



\subsection*{(c)}



\subsection*{(d)}



\subsection*{(e)}



\subsection*{(f)}



\subsection*{(g)}



}

\end{document}