\documentclass{article}
\linespread{1.3}
\usepackage[margin=50pt]{geometry}
\usepackage{amsmath, amsthm, amssymb, amsthm, tikz, fancyhdr}
\pagestyle{fancy}
\renewcommand{\headrulewidth}{0pt}
\newcommand{\changefont}{\fontsize{15}{15}\selectfont}

\fancypagestyle{firstpageheader}
{
  \fancyhead[R]{\changefont Michael Huang \\ CFRM 420 \\ Homework 1}
}

\begin{document}

\thispagestyle{firstpageheader}

\section*{1.}
{\Large 

\subsection*{(a)}

Plot the daily arithmetic returns.

\begin{verbatim}
  Text enclosed inside \texttt{verbatim}
  environment 
  is printed directly 
  and all \LaTeX{} commands are ignored.
\end{verbatim}

\subsection*{(b)}
Use
\begin{verbatim} to.monthly(msft.prices, OHLC=FALSE) \end{verbatim}
to obtain the monthly adjusted stock prices, and calculate the monthly log returns for each month from February to August inclusive.


\framebox[1.1\width]{\textbf{answer}}

}

\section*{2.}
{\Large

Suppose an investor has a portfolio of Microsoft and Apple stock. Use the Microsoft prices from Question 1 and obtain the Apple prices in a similar way. The investment period is 2021-01-04 to 2021-08-31 as before. The proportion of the portfolio invested in the Microsoft stock is 60\% at the start of the period. On 2021-05-03, the investor rebalances the portfolio so that the proportion invested in Apple stock is 60\%. Find the arithmetic and log return at the end of the period.

}

\section*{3.}
{\Large 

Consider the following situations. \\
An asset has a 20\% return in the first month and a -20\% return in the second month.
An asset has a -20\% return in the first month and a 20\% return in the second month.
Which situation is better for an investor in the asset if the returns are arithmetic returns? What if they were log returns? Explain your answer.

}

\end{document}